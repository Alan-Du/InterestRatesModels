\newcommand{\pluginName}{Hull-White one factor model}
\newcommand{\pluginVersion}{1.0.3}

\input{../../../DocumentationTemplate/TemplateL3}

\section{Introduction}
This plug-in implements Hull and White one factor models. For a general reference on this model see \cite{Hull:OptFutDer}.

\section{How to use the plug-in}

\begin{itemize}
  \item In the Fairmat user interface when you create a new stochastic process you will find the additional option ``Hull-White (one factor)''.
  \item The stochastic process is defined by the parameters shown in the following table.
\item The stochastic process is defined by the parameters shown in the following table.

\begin{center}
\begin{tabular}{|l|c|}
	\hline
	\textbf{Fairmat}&\textbf{Documentation}\\
	\textbf{notation}&\textbf{notation}\\
	\hline
	alpha & $\alpha$\\
	sigma & $\sigma$\\
	\hline
\end{tabular}
\end{center}
\end{itemize}

\section{Implementation Details}

\subsection{Process dynamics}

The Hull-White one factor model is used to describe the evolution of the short rate. It is defined by the following stochastic differential equations
\begin{equation}
dr(t) = (\theta(t)-\alpha r(t))dt + \sigma dW
\end{equation}
where $r$ is the short rate process.

The parameters have the following interpretation
\begin{itemize}
\item $\alpha$ is related to the short rate mean reversion speed, the greater its value, the stronger is the mean reversion
\item $\sigma$ is the short rate volatility
\end{itemize}

\subsection{Fitting the model to initial yield curve}

If $F(0,T)$ is the function of instantaneous forward rate, that is
\begin{equation}
F(0,T) = -\displaystyle \frac{\partial}{\partial T}\ln[P(0,T)]
\end{equation}

it can be shown that in the Hull-White model it is given by

\begin{equation}
F(0,T) = e^{-\alpha T}r(0) + \int_0^T e^{-\alpha (T-u)}\theta(u)du - \frac{\sigma^2}{2\alpha^2}\left[1-e^{-\alpha T}\right]^2
\end{equation}

so that $\theta(T)$ can be written as

\begin{equation}
\theta(T) = \alpha F(0,T)+\frac{\partial}{\partial T}F(0,T)+\frac{\sigma^2}{2\alpha}\left[1-e^{-2\alpha T}\right]
\end{equation}

The function $\theta(T)$ is then calculated from the initial zero rate yield curve.

\subsection{Solution to the SDE, mean and variance of the process}

Knowing the process at time $t$, the process at time $t+\tau$ is given by
\begin{align}
r(t+\tau) & = r(t) e^{-\alpha \tau} + e^{-\alpha (t+\tau)}\int_t^{t+\tau}\!\!\!\!e^{\alpha u}\theta(u)du + \sigma e^{-\alpha (t+\tau)}\int_t^{t+\tau}\!\!\!\!e^{\alpha u}dW(u)\\
 & = r(t) e^{-\alpha \tau} + a(t+\tau) - e^{-\alpha \tau} a(t) + \sigma e^{-\alpha (t+\tau)}\int_t^{t+\tau}\!\!\!\!e^{\alpha u}dW(u) \label{eq:HWevol}
\end{align}
where we have used the formula relating $\theta(T)$ and $F(0,T)$ and where the function $a(t)$ is defined as
\begin{equation}
a(t) = F(0,t) + \frac{\sigma^2}{2\alpha^2}\left[1-e^{-\alpha t}\right]^2
\end{equation}

From these formulas we know that $r(t)$ is normally distributed and we can calculate the expected value and the variance for $r$.

\begin{eqnarray}
\mathbb{E}[r(t+\tau)|r(t)] & = & r(t) e^{-\alpha \tau} + a(t+\tau) - e^{-\alpha\tau}a(t) \\
\mathrm{Var}[r(t+\tau)|r(t)] & = & \frac{\sigma^2}{2\alpha}\left[1-e^{-2\alpha\tau}\right]
\end{eqnarray}

Defining $t_1=t$ and $t_2=t+\tau$, formula \ref{eq:HWevol} can be rewritten as
\begin{align}
r(t_2) & = r(t_1) e^{-\alpha (t_2-t_1)} + a(t_2) - e^{-\alpha(t_2-t_1)}a(t_1) + \sigma e^{-\alpha t_2}\int_{t_1}^{t_2}e^{\alpha u}dW(u)\nonumber\\
  & = r(t_1) e^{-\alpha (t_2-t_1)} + F(0,t_2) - e^{-\alpha (t_2-t_1)}F(0,t_1) + \nonumber\\
  & \phantom{mmmm} + \frac{\sigma^2}{\alpha^2}e^{-\alpha t_2}\left[\cosh(\alpha t_2)-\cosh(\alpha t_1)\right] + \sigma e^{-\alpha t_2}\int_{t_1}^{t_2}e^{\alpha u}dW(u)
\end{align}
which is useful in simulating Monte Carlo path for Hull-White model, in fact $r(t_2)$ is distributed as
\begin{align}
r(t_2) \sim\  & r(t_1) e^{-\alpha (t_2-t_1)} + F(0,t_2) - e^{-\alpha (t_2-t_1)}F(0,t_1) + \nonumber\\
 & \phantom{mmmm}+ \frac{\sigma^2}{\alpha^2}e^{-\alpha t_2}\left[\cosh(\alpha t_2)-\cosh(\alpha t_1)\right] + \nonumber\\
 & \phantom{mmmm}+ \frac{\sigma}{\sqrt{2\alpha}}\sqrt{1-e^{-2\alpha(t_2-t_1)}}N(0,1).
\end{align}

\subsection{Bond Price formula}

The price of a zero coupon bough at time $t$ which pays a unit amount at time $T$ is given by

\begin{equation}
P(t,T) = A(t,T) e^{-B(t,T)r(t)}
\end{equation}
where
\begin{align}
& B(t,T) = \frac{1}{\alpha}\left[ 1 - e^{-\alpha(T-t)} \right]\\
& A(t,T) = \frac{P(0,T)}{P(0,t)}\exp\left[ -B(t,T) \frac{\partial \ln[P(0,t)]}{\partial t} \right.\nonumber\\
& \qquad\qquad\qquad\qquad\qquad\qquad\qquad \left. - \frac{\sigma^2}{4\alpha^3}\left( e^{-\alpha T} - e^{-\alpha t} \right)^2 \left( e^{2\alpha t} -1 \right) \right]
\end{align}

\section{Calibration}
The plug-ins contains also an extension for calibrating Hull and white one factor model from a matrix of cap volatilities.
\subsection{Objective Function}
The estimator tries to minimize the differences between the Black-cap prices and the prices of caps calculated through the model. More explicitly the optimization uses the following objective function
\begin{equation}
\displaystyle\sum_{i=1}^n {\Big( {HW1^i(\alpha_1,\sigma_1)} - {Black}^i \Big)}^2  \, ,
\end{equation}
where ${HW1^i(\alpha_1,\sigma_1)}$ is the price of the $i^{th}$-cap by ``one factor H\&W'' model, ${Black}^i$ the price of the $i^{th}$~cap by the Black model, and $n$ the number of all caps into the Cap-Volatility matrix.
\subsection{Data for calibration}
The cap based calibration uses only some specific fields of the file ``Interest~rate~market~data~xml''. In particular:
\begin{itemize}
\item[\textbf{Market}:] a string describing the market to which data refer. Possible choices are
	\begin{itemize}
	\item[\textbf{EU}] for Europe
    \item[\textbf{US}] for USA
    \item[\textbf{UK}] for United Kingdom
    \item[\textbf{SW}] for Switzerland
    \item[\textbf{GB}] for United Kingdom
    \item[\textbf{JP}] for Japan
	\end{itemize}
\item[\textbf{Date}:] is the date to which data refer, the format is ddmmyyyy
\item[\textbf{ZRMarket}:] vector with zero coupon rates (continuously compounded rates)
\item[\textbf{ZRMarketDates}:] vector of maturities corresponding to ZRMarket
\item[\textbf{CapTenor}:] is the year fraction between two payments of the cap options
\item[\textbf{CapMaturity}:] vector of cap maturities (i.e. the rows of the following CapVolatility matrix)
\item[\textbf{CapRate}:] vector of cap strikes (i.e. the columns of the following CapVolatility matrix)
\item[\textbf{CapVolatility}:] matrix of cap-volatilities, i.e. the element CapVolatility$[ i , j ]$ is the Black model volatility of cap with maturity equal to CapMaturity$[ i ]$ and strike equal to CapRate$[ j ]$
\end{itemize}
An example of an ``Interest~rate~market~data~xml'' containing only these fields is the following
{\footnotesize{
\begin{verbatim}
<InterestRateMarketData>
<Market>EU</Market>
<Date>31122010</Date>
<ZRMarket>0.012 0.013 0.015 0.019 0.021 0.023 0.024 0.026 0.027 0.029 0.026</ZRMarket>
<ZRMarketDates>1 2 3 5 6 7 8 9 10 30 50</ZRMarketDates>
<CapTenor>0.5</CapTenor>
<CapMaturity>1 2 3 4 5 6 7 8 9 10 12 15 20</CapMaturity>
<CapRate>0.0175 0.02 0.0225 0.025 0.03 0.035 0.04 0.05 0.06 0.07 0.08 0.1</CapRate>
<CapVolatility>0.0000	    0.5262	    0.4714	    0.5135	    0.5417	    0.5208	    0.5551	    0.5698	    0.5754	    0.5857	    0.5845	    0.5984
	0.0000	    0.5180	    0.5490	    0.4939	    0.5139	    0.5183	    0.5235	    0.5311	    0.5314	    0.5404	    0.5433	    0.5429
	0.0000	    0.4719	    0.4762	    0.4695	    0.4746	    0.4763	    0.4626	    0.4689	    0.4683	    0.4767	    0.4770	    0.4784
	0.0000	    0.4514	    0.4437	    0.4355	    0.4220	    0.4341	    0.4153	    0.4225	    0.4210	    0.4317	    0.4289	    0.4402
	0.0000	    0.4194	    0.4099	    0.3993	    0.3829	    0.3833	    0.3687	    0.3703	    0.3734	    0.3775	    0.3810	    0.3909
	0.0000	    0.4013	    0.3854	    0.3712	    0.3491	    0.3455	    0.3303	    0.3303	    0.3344	    0.3345	    0.3421	    0.3477
	0.0000	    0.3839	    0.3646	    0.3468	    0.3222	    0.3188	    0.3025	    0.2932	    0.2972	    0.3002	    0.3135	    0.3198
	0.0000	    0.3684	    0.3524	    0.3395	    0.3034	    0.2962	    0.2769	    0.2773	    0.2731	    0.2774	    0.2830	    0.2925
	0.0000	    0.3559	    0.3387	    0.3276	    0.2911	    0.2817	    0.2651	    0.2601	    0.2557	    0.2608	    0.2722	    0.2766
	0.0000	    0.3474	    0.3336	    0.2607	    0.2780	    0.2393	    0.2508	    0.2424	    0.2439	    0.2476	    0.2579	    0.2652
	0.0000	    0.3367	    0.3124	    0.2994	    0.2669	    0.2496	    0.2344	    0.2229	    0.2283	    0.2340	    0.2372	    0.2437
	0.0000	    0.3173	    0.2977	    0.2643	    0.2528	    0.2361	    0.2204	    0.2074	    0.2138	    0.2156	    0.2256	    0.2357
	0.0000	    0.2992	    0.2889	    0.2585	    0.2369	    0.2285	    0.2072	    0.2025	    0.2028	    0.2005	    0.2117      0.2211</CapVolatility>
</InterestRateMarketData>
\end{verbatim}
}}

\bibliographystyle{unsrt}
\bibliography{../../../DocumentationTemplate/bibliography}

\end{document}
